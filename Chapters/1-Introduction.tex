\section{Datenformat}

\cite{default}

\noindent Beschreibung des Städte-Graphen für das Travelling-Salesman-Problem:

\begin{itemize}
  \item Header. Name und Problemtyp, Kommentar, Dimension des Problems (und der zugehörigen Matrix), Format der Kanten-Gewichte (volle Matrix)
  \item Kantengewichte. 29 x 29 Matrix mit den Reisekosten zwischen den einzelnen Städten, symmetrische Matrix => ungerichteter Graph
  \item Positionen der Städte. Liste mit Stadt-Nummer und X-, Y-Koordinaten
\end{itemize}


\section{Testläufe mit tspVorlage.m}

\begin{tabular}{ | r | c | c | }
  \hline
  \# & Best Objective value & in Generation \\
  \hline
  1  & 3324 & 288 \\
  2  & 3041 & 345 \\
  3  & 3180 & 388 \\
  4  & 3148 & 244 \\
  5  & 3102 & 372 \\
  6  & 2898 & 243 \\
  7  & 2980 & 287 \\
  8  & 2764 & 139 \\
  9  & 2982 & 307 \\
  10 & 3490 & 319 \\
  11 & 2974 & 293 \\
  12 & 3281 & 364 \\
  13 & 2991 & 313 \\
  14 & 2967 & 261 \\
  15 & 3047 & 353 \\
  \hline
\end{tabular}


\section{Parameter und Optionen}

\begin{lstlisting}
geaoptset()
VariableFormat',            5 ...         % Use permutation variables  
NumberSubpopulation',   1 ...         % Number of subpopulation
NumberIndividuals',       50 ...        % Number of individuals per subpopulation
Selection.Name',          'selrws' ...  % Define the selection function
Selection.GenerationGap',   1 ...         % Set generation gap to 100%   
Recombination.Name',    'recpm' ...   % Define the recombination function
Mutation.Name',           'mutswap' ...   % Define the mutation functions
Mutation.Rate',           10 ...        % Define the mutation rate                   

Output.TextInterval',     100 ...       % Text output every 100 generations
Output.GrafikInterval',   100 ...       % Grafic results every 100 generations
Output.StatePlotInterval',  100       ...
Output.StatePlotFunction',  'plottsplib' ...
Termination.MaxGen',    400 ...       % Terminate after xx generations
Termination.Method',    1 ...         % Termination method to use

Output.SaveTextInterval',   0 ...                 % Text to File every xx generations
Output.SaveTextFilename', [FileNameBase '.txt'] ... % Filename of result file, absolut or relative path may be included
Output.SaveBinDataInterval',  0 ...                 % Binary Data to File every xx generations
Output.SaveBinDataFilename', [FileNameBase '.mat'] ... % Filename of binary file, absolut or relative path may be included


System.ObjFunFilename',   'objtsplib'
GeaOpt = geaoptset( GeaOpt , 'System.ObjFunAddPara', {TSPLIB_NAME});
% Get variable boundaries from objective function
   VLUB = geaobjpara(GeaOpt.System.ObjFunFilename, 1, GeaOpt.System.ObjFunAddPara)
   GeaOpt = geaoptset( GeaOpt , 'System.ObjFunVarBounds', VLUB); VLUB = [];

% Call main GEA function
   [xnew, GeaOpt] = geamain2(objfun, GeaOpt, VLUB, []);
\end{lstlisting}


\section{Automatisierte Testläufe}

Ergebnis von [xnew, GeaOpt] = geamain2()
xnew(1,:) liefert den besten Weg
GeaOpt.Run.BestObjectiveValue liefert die geringsten Kosten

\subsection{Ergebnisse}

\begin{table}[tbph]
\begin{tabular}{ | c || r | r | r | r | r | }
\hline
Parameter & \# Iterationen & Mittelwert & Std.-Abw. & Laufzeit & Min, Max \\
\hline
   1.00 &  10 & 2515.70 &   83.09 &  127.44 & 2418, 2729 \\
   2.00 &  10 & 2650.80 &   55.88 &  128.35 & 2566, 2779 \\
   5.00 &  10 & 2814.90 &   40.74 &  129.43 & 2745, 2886 \\
  10.00 &  10 & 2956.10 &   66.79 &  129.32 & 2844, 3067 \\
  20.00 &  10 & 2942.30 &   59.01 &  130.21 & 2789, 3001 \\
  30.00 &  10 & 2918.00 &   88.64 &  129.95 & 2723, 3020 \\
\hline
\end{tabular}
\caption{Migration.Interval}\label{Migration.Interval}
\end{table}


\begin{table}[tbph]
\begin{tabular}{ | c || r | r | r | r | r | }
\hline
Parameter & \# Iterationen & Mittelwert & Std.-Abw. & Laufzeit & Min, Max \\
\hline
   0.01 &  10 & 2956.50 &   66.98 &  127.93 & 2858, 3030 \\
   0.02 &  10 & 2932.90 &   62.62 &  128.32 & 2823, 3054 \\
   0.05 &  10 & 2908.90 &   77.54 &  129.81 & 2789, 3015 \\
   0.10 &  10 & 2977.10 &   57.18 &  130.47 & 2912, 3079 \\
   0.15 &  10 & 2930.80 &   80.85 &  129.96 & 2759, 3043 \\
   0.25 &  10 & 2916.10 &   70.82 &  130.00 & 2761, 3021 \\
\hline
\end{tabular}
\caption{Migration.Rate}\label{Migration.Rate}
\end{table}


\begin{table}[tbph]
\begin{tabular}{ | c || r | r | r | r | r | }
\hline
Parameter & \# Iterationen & Mittelwert & Std.-Abw. & Laufzeit & Min, Max \\
\hline
   0.00 &  10 & 2938.20 &   63.43 &  131.35 & 2830, 3033 \\
   1.00 &  10 & 2914.40 &   65.73 &  129.25 & 2816, 3038 \\
\hline
\end{tabular}
\caption{Migration.Selection}\label{Migration.Selection}
\end{table}


\begin{table}[tbph]
\begin{tabular}{ | c || r | r | r | r | r | }
\hline
Parameter & \# Iterationen & Mittelwert & Std.-Abw. & Laufzeit & Min, Max \\
\hline
   0.00 &  10 & 2962.40 &   78.42 &  131.27 & 2813, 3078 \\
   1.00 &  10 & 2945.30 &   89.98 &  130.39 & 2776, 3044 \\
   2.00 &  10 & 2963.20 &   67.01 &  131.35 & 2835, 3044 \\
\hline
\end{tabular}
\caption{Migration.Topology}\label{Migration.Topology}
\end{table}


\begin{table}[tbph]
\begin{tabular}{ | c || r | r | r | r | r | }
\hline
Parameter & \# Iterationen & Mittelwert & Std.-Abw. & Laufzeit & Min, Max \\
\hline
  10.00 &  11 & 3604.36 &  156.64 &   -1.00 & 3271, 3838 \\
  20.00 &  11 & 3331.55 &  194.16 &   -1.00 & 3096, 3657 \\
  30.00 &  10 & 3235.10 &  174.55 &   -1.00 & 2960, 3569 \\
  50.00 &  10 & 3090.30 &  151.34 &   -1.00 & 2829, 3395 \\
  75.00 &  10 & 3005.70 &   88.17 &   -1.00 & 2851, 3181 \\
 100.00 &  10 & 2985.00 &   95.18 &   -1.00 & 2800, 3123 \\
 200.00 &  10 & 2990.50 &   90.44 &   -1.00 & 2862, 3181 \\
 300.00 &  10 & 2926.10 &   84.25 &   -1.00 & 2716, 3024 \\
 400.00 &  10 & 2925.10 &   59.14 &   -1.00 & 2838, 3020 \\
 500.00 &  10 & 2881.20 &   54.06 &   -1.00 & 2810, 2962 \\
 625.00 &  10 & 2872.10 &   81.27 &   -1.00 & 2772, 3028 \\
 750.00 &  10 & 2897.20 &   71.38 &   -1.00 & 2746, 2988 \\
1000.00 &  10 & 2858.20 &   85.99 &   -1.00 & 2694, 2987 \\
1250.00 &  10 & 2832.30 &   43.77 &   -1.00 & 2764, 2897 \\
1500.00 &  10 & 2847.10 &   78.61 &   -1.00 & 2648, 2963 \\
\hline
\end{tabular}
\caption{NumberIndividuals}\label{NumberIndividuals}
\end{table}


\begin{table}[tbph]
\begin{tabular}{ | c || r | r | r | r | r | }
\hline
Parameter & \# Iterationen & Mittelwert & Std.-Abw. & Laufzeit & Min, Max \\
\hline
   1.00 &  10 & 3080.00 &  103.00 &   -1.00 & 2913, 3232 \\
   2.00 &  10 & 2993.90 &  173.72 &   -1.00 & 2695, 3297 \\
   3.00 &  10 & 2974.30 &   63.59 &   -1.00 & 2838, 3045 \\
   5.00 &  10 & 2916.70 &   99.34 &   -1.00 & 2779, 3068 \\
   7.00 &  10 & 2915.60 &   55.28 &   -1.00 & 2828, 2984 \\
  10.00 &  10 & 2878.90 &   72.28 &   -1.00 & 2719, 3004 \\
  15.00 &  10 & 2854.50 &   58.09 &   -1.00 & 2780, 2958 \\
  20.00 &  10 & 2867.50 &   93.64 &   -1.00 & 2689, 2974 \\
  30.00 &  10 & 2808.30 &   57.47 &   -1.00 & 2700, 2894 \\
  40.00 &  10 & 2836.00 &   71.85 &   -1.00 & 2711, 2924 \\
  50.00 &  10 & 2809.10 &   49.99 &   -1.00 & 2742, 2876 \\
\hline
\end{tabular}
\caption{NumberSubpopulation}\label{NumberSubpopulation}
\end{table}



