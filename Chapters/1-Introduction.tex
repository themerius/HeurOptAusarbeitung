\section{Datenformat}

\cite{default}

\noindent Beschreibung des Städte-Graphen für das Travelling-Salesman-Problem:

\begin{itemize}
  \item Header. Name und Problemtyp, Kommentar, Dimension des Problems (und der zugehörigen Matrix), Format der Kanten-Gewichte (volle Matrix)
  \item Kantengewichte. 29 x 29 Matrix mit den Reisekosten zwischen den einzelnen Städten, symmetrische Matrix => ungerichteter Graph
  \item Positionen der Städte. Liste mit Stadt-Nummer und X-, Y-Koordinaten
\end{itemize}


\section{Testläufe mit tspVorlage.m}

\begin{tabular}{ | r | c | c | }
  \hline
  \# & Best Objective value & in Generation \\
  \hline
  1  & 3324 & 288 \\
  2  & 3041 & 345 \\
  3  & 3180 & 388 \\
  4  & 3148 & 244 \\
  5  & 3102 & 372 \\
  6  & 2898 & 243 \\
  7  & 2980 & 287 \\
  8  & 2764 & 139 \\
  9  & 2982 & 307 \\
  10 & 3490 & 319 \\
  11 & 2974 & 293 \\
  12 & 3281 & 364 \\
  13 & 2991 & 313 \\
  14 & 2967 & 261 \\
  15 & 3047 & 353 \\
  \hline
\end{tabular}


\section{Parameter und Optionen}

\begin{lstlisting}
geaoptset()
VariableFormat',            5 ...         % Use permutation variables  
NumberSubpopulation',   1 ...         % Number of subpopulation
NumberIndividuals',       50 ...        % Number of individuals per subpopulation
Selection.Name',          'selrws' ...  % Define the selection function
Selection.GenerationGap',   1 ...         % Set generation gap to 100%   
Recombination.Name',    'recpm' ...   % Define the recombination function
Mutation.Name',           'mutswap' ...   % Define the mutation functions
Mutation.Rate',           10 ...        % Define the mutation rate                   

Output.TextInterval',     100 ...       % Text output every 100 generations
Output.GrafikInterval',   100 ...       % Grafic results every 100 generations
Output.StatePlotInterval',  100       ...
Output.StatePlotFunction',  'plottsplib' ...
Termination.MaxGen',    400 ...       % Terminate after xx generations
Termination.Method',    1 ...         % Termination method to use

Output.SaveTextInterval',   0 ...                 % Text to File every xx generations
Output.SaveTextFilename', [FileNameBase '.txt'] ... % Filename of result file, absolut or relative path may be included
Output.SaveBinDataInterval',  0 ...                 % Binary Data to File every xx generations
Output.SaveBinDataFilename', [FileNameBase '.mat'] ... % Filename of binary file, absolut or relative path may be included


System.ObjFunFilename',   'objtsplib'
GeaOpt = geaoptset( GeaOpt , 'System.ObjFunAddPara', {TSPLIB_NAME});
% Get variable boundaries from objective function
   VLUB = geaobjpara(GeaOpt.System.ObjFunFilename, 1, GeaOpt.System.ObjFunAddPara)
   GeaOpt = geaoptset( GeaOpt , 'System.ObjFunVarBounds', VLUB); VLUB = [];

% Call main GEA function
   [xnew, GeaOpt] = geamain2(objfun, GeaOpt, VLUB, []);
\end{lstlisting}


\section{Automatisierte Testläufe}

Ergebnis von [xnew, GeaOpt] = geamain2()
xnew(1,:) liefert den besten Weg
GeaOpt.Run.BestObjectiveValue liefert die geringsten Kosten
