\section{Rekombination}\label{recombination}

Die Rekombination kreiert neue Chromosome durch eine Kombination zweier zufällig
ausgewählter bereits existierender Chromosome. Dies entspricht der normalen
Fortpflanzung in der Biologie.
Das in \emph{GEATbx} enthaltene \emph{Partially matched Crossover} Verfahren
wird im Folgenden mit dem \emph{Order Crossover} von Lawrence Davis verglichen.

% http://www.geatbx.com/docu/options-03.html
% http://citeseerx.ist.psu.edu/viewdoc/download?doi=10.1.1.61.6575&rep=rep1&type=pdf
% http://chern.ie.nthu.edu.tw/alg2003/chap-8-9.pdf
% http://siebn.de/download/GenetischeAlgorithmen.pdf

\subsection{Order Crossover (OX)}
Dieses von Lawrence Davis in \citep{ox} beschriebene Verfahren ist nicht im
Umfang von \emph{GEATbx} enthalten und muss daher selbst implementiert werden.

Beschreibung und Implementierung % TODO Code

\begin{figure}[h!]
  \centering
  \includegraphics[width=1.0\textwidth]{Figures/recox.pdf}
  \caption{Order Crossover (OX)}\label{fig.recox}
\end{figure}


\subsection{Partially matched Crossover (recpm)}
Beschreibung \citep{erben} % TODO

\begin{figure}[h!]
  \centering
  \includegraphics[width=0.7\textwidth]{Figures/recpm.pdf}
  \caption{Partially matched Crossover (recpm)}\label{fig.recpm}
\end{figure}


\subsection{Ergebnisse}

% TODO
\input{Chapters/gen/Recombination.Name}
Entscheidung: recox

