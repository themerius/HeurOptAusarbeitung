\section{Automatisiertes Testsystem}\label{testsystem}

In der Aufgabenstellung \citep{aufg} ist verlangt \emph{eine Konstellation zu finden,
unter der der Algorithmus möglichst gute Lösungen in möglichst kurzer Zeit} findet.
Um dies zu bewerkstelligen, müssen bessere Parametrisierungen für den Algorithmus
gefunden werden. Dazu wurde ein automatisiertes Testsystem aufgebaut, um
die Parameter systematisch abzusuchen, sodass ideale Parametrisierungen
gefunden werden. Das Testsystem wird im Folgenden erklärt, die einzelnen Testläufe
und deren Ergebnisse werden in den darauffolgenden Kapiteln präsentiert und
analysiert.

Neben dem \emph{GEATbx}-Framework besteht das automatisierte Testsystem aus den
folgenden drei Matlab-Funktionen beziehungsweise -Skripten.


\paragraph{tsp.m} enthält die Matlab-Funktion {\tt tsp()}, welche auf dem
zur Verfügung gestellten Skript \emph{tspVorlage.m} basiert. Der folgende
Codeabschnitt enthält die formale Schnittstelle der Funktion, wobei \emph{name}
und \emph{value} den zu testenden Parameter des evolutionären Algorithmus mit
zugehörigen Wert bezeichnen.
Für manche Testläufe müssen weitere Parameter gesetzt werden, beispielsweise
bei den Migrationsparametern in Kapitel \ref{populations}. Dafür dient der
dritte Parameter \emph{args}, welcher als Liste in der Form
\{ Key$_{1}$, Value$_{1}$, Key$_{2}$, Value$_{2}$, ..., Key$_{N}$, Value$_{N}$ \}
beliebig viele weitere Schlüssel-Wert-Paare an den Algorithmus übergibt.
Die Rückgabewerte entsprechen exakt denen der Funktion {\tt geamain2()} des
\emph{GEATbx}-Frameworks.

\lstinputlisting[language=Matlab, lastline=1]{Code/tsp.m}

\noindent Die Initialisierung mit Default-Werten entspricht genau der Datei
\emph{tspVorlage.m} mit dem Unterschied, dass der Dateiname fest auf "'bays29"'
gesetzt ist. Vor dem Aufruf von {\tt geamain2()} werden schließlich die
übergebenen Parameter in die GeaOpt-Struktur gesetzt:
\lstinputlisting[language=Matlab, firstnumber=78, firstline=78, lastline=85]{Code/tsp.m}


\paragraph{testTsp.m} stellt ebenfalls eine Matlab-Funktion dar, welche
verschiedene Werte für einen Parameter des evolutionären Algorithmus durchtestet
und die jeweiligen Ergebnisse der einzelnen Testläufe in einer Textdatei
protokolliert, um diese später statistisch auswerten zu können.

\lstinputlisting[language=Matlab, lastline=1]{Code/testTsp.m}

\noindent Das Argument \emph{name} bezeichnet den Parameter, welcher mit
unterschiedlichen Werten getestet werden soll, welche in Form einer Liste als
\emph{values} übergeben werden.
Um für die statistische Auswertung einen guten Mittelwert und eine plausible
Aussage über die Standardabweichung treffen zu können, lässt sich mittels
\emph{runs} die Anzahl der Testläufe pro Wert spezifizieren.
Falls die Funktion {\tt tsp()} wie oben beschrieben mit weiteren Schlüssel-Wert-Paaren
parametrisiert werden muss, geschieht dies über den \emph{varargin}-Mechanismus,
welcher es beim Aufruf von {\tt testTsp()} erlaubt, beliebig viele Argumente zu
definieren und diese automatisch in eine Liste umwandelt. So sind Aufrufe in der
folgenden Form möglich:

\begin{lstlisting}[language=Matlab]
testTsp('Key1', {0, 1, 2}, 10, 'Key2', value2, 'Key3', value3);
\end{lstlisting}

\noindent Im folgenden Abschnitt ist der Ablauf des Tests mit Aufruf von
{\tt tsp()} inklusive einer Zeitmessung in gekürzter Form dargestellt:

\lstinputlisting[language=Matlab, firstnumber=8, linerange={8-8,11-12,14-18,34-35}]{Code/testTsp.m}

\noindent Zum Speichern der Ergebnisse dienen die folgenden Zeilen Code.
Der Dateiname wird aus dem Parameternamen und dem Wert zusammengesetzt. Um für
bestimmte Tests eine Eindeutigkeit zu gewährleisten, lässt sich zudem über eine ungerade
Anzahl an Argumenten in \emph{varargin} eine zusätzliche Zeichenkette in den
Dateinamen einfügen (es wird das letzte Element der Liste verwendet).
Gespeichert wird neben den Kosten des kürzesten Weges der kürzeste Weg selbst
und die vom evolutionären Algorithmus benötigte Zeit.

\lstinputlisting[language=Matlab, firstnumber=21, firstline=21, lastline=33]{Code/testTsp.m}


\paragraph{suite.m} ist ein Matlab-Skript und enthält alle durchzuführenden Tests
als Aufrufe der Funktion {\tt testTsp()}. Die einzelnen Aufrufe werden zum Teil
in den folgenden Kapiteln dargestellt.


\paragraph{stats.py} ist ein Python-Skript, welches die von der Funktion
{\tt testTsp()} generierten Dateien einliest und Mittelwert und Standardabweichung
über die verschiedenen Testläufe berechnet.
Das Skript generiert zu jedem Parameter direkt eine \LaTeX-Tabelle mit den
zugehörigen Werten, welche in den folgenden Kapiteln direkt eingefügt sind.

